■ デジタルサイネージと実世界GUI

駅や街角でデジタルサイネージを見かける機会が多くなってきた。
遅延のような情報の伝達や広告のためにサイネージは効果的に利用されているが、
その可能性が充分に生かされているとはいえないのが現状である。

特に、個々のユーザに対応した情報のやりとりが全くできないのは大きな問題である。
昔の駅には「掲示板」が用意されているのが普通であり、
駅の利用者同志の非同期的なコミュニケーションに活用されていた。
現在は携帯電話が普及したためか駅の掲示板のほとんどは撤去されてしまっているが、
駅という特定の場所において時間差のあるコミュニケーションを行なうことができたという点で
掲示板は有用なものであった。
現在のデジタルサイネージを利用して同様のサービスを提供することは充分可能なはずであるが、
そのような目的のためにサイネージは現在利用されていない。

利用者が能動的的なコミュニケーションを行なわない場合でも、
利用者にとって有益な情報をサイネージで個別に提供することができれば有益なはずであるが、
現在のサイネージはテレビと同じように誰にでも同じ情報を提供しているだけであり、
特別な要求や検索に答えるような工夫がされていないのが現状である。
ショッピングモールのサイネージでマップを見ることはできるかもしれないが、
利用者の性別や年齢により表示を変えることすら行なわれていない。
サイネージを利用するとき、個人的な情報を少しでも利用することができれば
利用価値が大きく変わる可能性がある。

サイネージのインタフェースはパソコンや携帯電話や家電機器のインタフェースと
全く異なるのが普通なので、検索のような機能が用意されているサイネージであっても
利用者は簡単に操作することはできないし、
長い間立ち止まって操作に習熟しようとする利用者もほとんどいないと思われる。
現在のサイネージは
設置場所固有の情報とインターネット上の通常の情報の両方を扱うことができるので、
インターネット上の情報にしかアクセスできない携帯電話のような装置を利用する場合よりも
有益な情報を提供できてしかるべきであるが、
操作の難しさのため、
利用者はもっぱら個人のパソコンや携帯電話を利用しているのが現状である。


個人が情報を得たり情報を発信したりする役にたっていない

駅の掲示板というのがあったが、最近は見られない
ケータイのせいだと思われるが、デジタルサイネージがあればそれと同等以上の効果が本当はあるはず

また、個人と場所に適応した情報を伝えることもできるはずだがそれもできていない
テレビを見るのと同じであり、とても有効利用されていない

唯一自動販売機というのはあるが、これも特にデジタルサイネージだから有用ということになっていない

特定の公共の場所で使えるのだからもっと有用であるべき

ネットはどこにでもあるし
個人情報を示すSuicaもあるのに有効利用されていないのは、
アイデアが足りないからであろう

ユーザが慣れていない点もある
サイネージの前でおういう操作をすればいいのかわからない
改札のSuicaですら浸透するには時間がかかったのだから
サイネージの操作など難しいのはあたりまえかもしれない

■

パソコンの操作は結構誰でも慣れている

ブラウザの使い方はかなりの人が理解しているだろうし

パソコンのファイル操作や検索なども多くの人が使えるはずである

しかしサイネージはパソコンじゃないから使えないのである

じゃぁサイネージをパソコンと同じにすればいいかというとそうはいかない
キーボードが使えないし、ログインが必要かもしれない
あらゆるサイネージにマウスやキーボードを用意しておくことはできないだろう

つまり、実世界においてパソコン操作的なものを可能にするハードウェア/ソフトウェアが無いのが問題なのだろう

「実世界でのGUI」があれば良いのである


ちなみにこれは技術的な問題ではない。

イディオムがまだ発明されていないことと
ユーザの意識がそれに対応していないことが原因である

ユーザが「そもそも」何をしたいかを簡単にサイネージ上で表現することができれば話は全然かわる

■ 実世界GUIの実現

公共のサイネージにおいて、家でパソコンを扱うのと同じようにブラウザや地図を操作できたら
便利なことだろう。

しかも、自分の地図や自分のブックマークがそこで見られるならとても便利だろう。

現在のサイネージは、
有用な情報を提供してはいるものの、
「トイレはどこかみたいに」
誰にとっても同じ情報しか提供してないし、
その情報にアクセスする手段も独自のものになっている。
それでは使えないのはあたりまえである。

「個人適応」と、「使いなれた利用法」を提供することが重要であろう


実はこれは簡単にできる。
Suicaやスマホを使えばいいのである。

SuicaにはRFIDが入ってありから、Suicaを認識できるサイネージの上でSuicaを動かせば
マウスのように使えるはずだし、
SuicaのIDから個人の嗜好などを利用できる


しかしSuicaリーダをあちこちに置くのは実際的ではない
あらゆるポスターにリーダを置くのは大変だろう
しかし最近のドコモなどのRFIDリーダ入りケータイを使うとこの問題が完全に解決される。

ポスターやサイネージの方にSUicaやRFIDを貼っておき、
スマホでそれにタッチした後で様々な操作を行なうようにすればよい。



我々はこういうシステムを長年研究している。
10年前はただの研究だったが、現在はインフラが整ってきたので技術的には完全に実用段階になっているといえる。

問題はユーザの思いこみと、インタラクションのアイデアの実装だけなのである


■ サイネージ+実世界GUiで広がる世界

どんなことができるか列挙してみたい。

* 地図を見る
  ルートを教えてくれる
  そのルートに問題がないか、とか
  これは個人情報がわからないと案内不可能である
* 自分の趣味のものが近くにあるかどうか
  買おうと思ってた本が駅の本屋にあることがわかる
* 好きなジャンルのレストランを調べる
  ついでにその状況もわかる
* 普通にググる
  ググるにはキーボードが必要と思われるかもしれないがそうでもない

■ ソフトウェアのインフラ

我々はGoldFishというインフラを作っている。

GOldFishはAndroidのアプリケーションで、
読んだIDのURLにジャンプするという単純なものである

しかし、Androidのセンサを利用することができるので
以下のようなことができる

---
---

これを利用するためには、サーバにJSを用意するだけでよい。
Android端末でRFIDにタッチすると
サーバ上のJavaScriptが端末上で動くわけだから、
端末に様々なプログラムをインストールしておく必要はないのである。


つまり、RFIDを貼ったポスタとサーバ上のプログラムだけ用意しておけば
自由自在に実世界GUIやそれを利用したサービスを提供できることになる。

■ 

GoldFIshの枠組みやRFIDリーダ入りケータイはまだ出たところであるが
その可能性は無限であり、
「パソコンを何に使うのか」と聞いてるみたいなものだといえるだろう。

早目にこういう可能性を考えることが大事だと思っている。

















