\documentclass[a4j,11pt,twocolumn]{jarticle}
% \usepackage{graphicx,layout,prosym2010}
% \usepackage{here}

\makeatletter

\def\section{\@startsection {section}{1}{\z@}{-3.5ex plus -1ex minus -.2ex}{2.3ex plus .2ex}{\normalsize\bf}}

\def\@maketitle{
  \newpage\null\vskip 1em%
  \begin{center}%
	\let\footnote\thanks
	{\Large \bf \@jtitle \par}
	{\Large \bf \@etitle \par}
	\vskip 1.5em
 	{\large \bf \begin{tabular}[t]{c}\@jauthor \end{tabular}\par}
	\vskip 1.0em
	%\begin{tabular}[t]{c} \@jcontact \end{tabular}\par
	%\vskip 1.5em%
	%{\unvbox\@jabstractbox}
	%\vskip2em
% English title
	%{\Large \bf \@etitle \par}
	%\vskip 1.5em
	{\large \begin{tabular}[t]{c}\@eauthor \end{tabular}\par}
	\vskip .5em
	{\small \begin{tabular}[t]{c}\@econtact \end{tabular}\par}
	\vskip 1.5em
	{\unvbox\@eabstractbox}\par
	\par \vskip .5em
  \end{center}
  {\small\bf キーワード: \@jkeywords}\par
  {\small\bf\sf Keywords: \@ekeywords}
  \par \vskip 1.5em
}
%%%%%%%%%%%%%%%%%%%%%%%%%%%%%%%%%%%%%%%%%%%%%%%%%
%
%	Definitions
%
% Japanese & English titles
%
\def\jtitle#1{\gdef\@jtitle{#1}}
\def\@jtitle#{\mkt\@warning\jtitle}
\def\etitle#1{\gdef\@etitle{#1}}
\def\@etitle{\mkt@warning\etitle}

\def\jkeywords#1{\gdef\@jkeywords{#1}}
\def\@jkeywords#{\mkt\@warning\jkeywords}
\def\ekeywords#1{\gdef\@ekeywords{#1}}
\def\@ekeywords{\mkt@warning\ekeywords}
%
% Japanese & English authors
%
\long\def\jauthor#1{\long\gdef\@jauthor{#1}}
\def\@jauthor{\mkt@warning\jauthor}
\long\def\eauthor#1{\long\gdef\@eauthor{#1}}
\def\@eauthor{\mkt@warning\eauthor}
%
%  Japanese & English contact address
%
\long\def\jcontact#1{\long\gdef\@jcontact{#1}}
\def\@jcontact{\mkt@warning\jcontact}
\long\def\econtact#1{\long\gdef\@econtact{#1}}
\def\@econtact{\mkt@warning\econtact}

% Japanese & English abstracts

\def\jabstractname{あらまし}
\def\eabstractname{}

\newbox\@jabstractbox \global\setbox\@jabstractbox\box\voidb@x
\newbox\@eabstractbox \global\setbox\@eabstractbox\box\voidb@x

\def\jabstract{%
   \global\setbox\@jabstractbox\vbox\bgroup
      \hsize\textwidth \@parboxrestore
      {\bfseries \jabstractname}  \parindent=1zw\relax}
\def\endjabstract{\egroup}

\def\eabstract{%
      \small
   \global\setbox\@eabstractbox\vbox\bgroup
      \hsize\textwidth \@parboxrestore
      {\bfseries \eabstractname}  \parindent=1zw\relax}
\def\endeabstract{\egroup}
%%%%%%%%%%%%%%%%%%%%%%%
\makeatother

\def\SP{スマートフォン}

\pagestyle{empty}

\long\def\comment#1{}

\begin{document}

\def\figwidth{7.5cm}

\jtitle{デジタルサイネージと実世界GUI}
\jauthor{増井 俊之 (慶應義塾大学 環境情報学部)}
\jkeywords{実世界インタフェース, 実世界GUI, デジタルサイネージ, {\SP}, RFID}
\ekeywords{real-world GUI, digital signage, smart phones, RFID}
\begin{eabstract}
Although various digital signage systems are set up in many train stations and
other public spaces, they are usually used for presenting general information
for the people around them,
and they usually cannot provide detailed information for individual users.
We propose controlling digital signages with people's smart phones
like controlling menus and scroll bars with a mouse on a personal computer.
We believe that the ``\textit{real-world GUI}'' will be the key to
practical use of digital signages in public spaces.
\end{eabstract}

\etitle{Digital Signage and Real-world GUI}
\eauthor{Toshiyuki Masui}
\econtact{Faculty of Environment and Information Studies, Keio University}

\maketitle

\section{デジタルサイネージの現状}

駅や街角でデジタルサイネージを見かける機会が多くなってきた。
電車の遅延情報の伝達や商品広告などのためにサイネージは効果的に利用されているが、
その可能性が充分に生かされているとは言い難い。

馴染みがない機械を利用するのは抵抗を感じるものなので
サイネージ利用のハードルは高い。
カードを改札機にタッチすれば改札扉が開くという
単純なインタフェースを持つSuicaの場合ですら、
誰もが使い方を理解するようになるまでには時間がかかったことを考えると、
誰もがサイネージを能動的に活用するようになるのは困難だと思われ、
現在のサイネージのほとんどはテレビのように情報を一方的に提供するだけにとどまっている。
サイネージのインタフェースはパソコンや携帯電話や家電機器のインタフェースとは異なるので、
検索のような機能が用意されているサイネージであっても気軽に操作できないし、
長い間立ち止まって操作の練習をしてもらうことは期待できない。

また、個々のユーザに対応した情報のやりとりができないことも問題である。
昔の駅には「掲示板」が用意されているのが普通であり、
利用者同志で時間差のあるコミュニケーションに活用されていた。
現在は携帯電話が普及したためか、
駅の掲示板のほとんどは撤去されてしまっているが、
駅という特定の場所において時間差のあるコミュニケーションを行なうことができたという点で
掲示板は有用なものであった。
現在のデジタルサイネージを利用して同様のサービスを提供することも可能なはずであるが、
そのような目的のためにサイネージは現在利用されていない。
利用者が情報を書き込むような能動的なコミュニケーションを行なわない場合でも、
利用者にとって有益な情報をサイネージで個別に提供することができれば有益なはずであるが、
現在のサイネージは
個々のユーザに対応した特別な要求や検索に答えるような工夫がされていないのが現状である。
ショッピングモールのサイネージでマップを見ることはできるかもしれないが、
利用者の性別や年齢により表示を変えることすら行なわれていない。

現在のサイネージは
設置場所固有の情報とインターネット上の通常の情報の両方を扱うことができるので、
インターネット上の情報にしかアクセスできない携帯電話のような装置を利用する場合よりも
有益な情報を提供できるはずであるが、
対話的に操作できるサイネージはほとんど存在しないし、
対話的なサイネージの操作は難しいと感じられるため、
利用者はもっぱら個人のパソコンや携帯電話を利用していると思われる。

個人的な情報を利用しつつ
簡単な操作でサイネージを活用することが可能であれば、
世の中のサイネージの利用価値が大きく向上する可能性がある。
% これはハードウェアやインフラの問題ではなく、
% 純粋にシステムの工夫が足りないことに起因している。
% 個人適応の機能及び使いなれた利用法を提供することが重要である。
長い時間をかけてパソコンのインタフェースがポピュラーになったのに比べるとサイネージはまだ歴史が浅く、
ユーザが戸惑うのは仕方がないともいえるが、
パソコンや携帯電話上の知見を利用したり
新しいセンサなどを工夫することによって、
サイネージが充分活用されていない問題が一気に解決する可能性がある。

\section{実世界GUI}

新しいシステムの使い方に習熟するのは難しいものであるが、
パソコンのように
ある程度馴染みのあるシステムに似たものであれば簡単に慣れることができるかもしれない。

パソコンは非常に複雑な機械であるが、
MacもWindowsもLinuxも
長年の共進化によって同じような操作性をもつようになっているため、
プルダウンメニューやスクロールバーのような
グラフィカルユーザインタフェース(GUI)部品の使い方に戸惑うユーザは少ないはずである。
一方、こういうものが標準化されていないサイネージでは、
はじめて使うユーザはどう使えばよいのかわからないうえに練習する機会も少ないため、
複雑な操作を行なうことはできないであろう。
%
サイネージの使い方がパソコンやブラウザの操作に似ていれば
多くの人がサイネージを簡単に使いこなせるようになる可能性がある。
極端にいえば、パソコンと全く同じ使い方でサイネージを利用できるのであれば
多くの人が今すぐサイネージを使うようになるであろう。
しかし、駅や街角に設置されるサイネージでは
パソコンと同じようなキーボードやマウスを利用することが難しいし、
そもそもパソコンは誰もがログインして使うようには設計されていない。
パソコンの利用法に似て異なる、
サイネージ用のインタフェース装置とインタフェース手法があれば良いと考えられる。

現在のパソコンは、
メニュー/アイコン/スクロールバーのような
GUIで操作するのが普通であり、
多くの人々はこれらの操作に慣れている。
サイネージにおいてもこのようなGUI部品を利用することができれば、
さほど混乱なくサイネージを活用できるようになると考えられる。
公共のサイネージにおいて、家でパソコンを扱うのと同じようにブラウザや地図を操作できれば便利であろう。
たとえば、自分の地図や自分のブックマークをサイネージですぐに参照できれば便利である。

パソコン上ではキーボードやマウスを使ってGUI部品を制御するが、
サイネージではこのような入力装置のかわりに
個人が持っている{\SP}を利用することが考えられる。
{\SP}は今後の普及が期待されるうえに、
傾きセンサや動きセンサを内蔵しているので、
マウスのような利用法が可能である。
また、{\SP}は独自のIDやユーザのデータを内蔵しているため、
操作者の嗜好や操作履歴を反映した処理を行なうことも容易である。

サイネージ上で{\SP}を活用して、
パソコン上のGUIに似た操作を行なう「実世界GUI」\cite{RealWorldGUI}を実現することによって、
サイネージの利用を劇的に改善することができると考えられる。

\section{実世界GUIの実現}

マウスは構造がシンプルで不正確な移動検出装置にすぎないが、
ソフトウェアの工夫により、
パソコン上で便利なGUIを利用できるようになっている。
実世界環境では正確なポインティングデバイスを
利用することはさらに難しいが、
実世界においてパソコン上のGUIに似た操作を行なうことは難しくない。

(1)マウスカーソルの下にボタンやメニューがあることを判定することができ、
(2)マウスの移動量にともなってカーソルを移動させることができれば、
パソコン上のほとんどのGUIを実現することができるのであるから、
これと同じことを実世界で実現できればよいことになる。

たとえば、実世界においてマウスのかわりにSuicaを利用することができる。
(1)ポスターの裏にSuicaリーダを内蔵しておき、それにSuicaをかざすことによってIDを読み取り、
(2)Suicaの動きを読みとる装置を用意しておけば、
Suicaをポスターにかざしてから動かすことによって
Suicaをマウスのように利用することができる。

沢山のSuicaリーダをポスターの裏に設置するのは実際的ではないが、
しかし最近のドコモなどから発売されている
Suicaを読み取り可能なNFCリーダ入り{\SP}を使うとこの問題が解決される。
この場合は、
(1){\SP}を何かにタッチしてIDを検出し、
(2){\SP}の動きを検出することができれば、
{\SP}を実世界でマウスのように利用することができる。
たとえば、
「曲選択」と書いたパネルを用意しておき、
その裏側にSuicaを貼っておけば、
(1)NFCリーダを内蔵した{\SP}で曲選択パネルにタッチすることによってSuicaのIDを読み取った後で、
(2){\SP}を移動すれば、
IDと移動量をもとにしてメニューのように{\SP}で選曲を行なうことができる。
また、このとき{\SP}がユーザの嗜好を知っていれば、
ユーザごとに異なる曲を選曲することができる。

我々はこういうシステムを長年研究している。
前者のようなシステムを\textit{FieldMouse}\cite{FieldMouse}、
後者のようなシステムを\textit{MouseField}\cite{MouseField}と呼んでおり、
様々な利用法を提案している。
10年ほど前この研究を開始したときは
まだNFCリーダつき{\SP}が存在しなかったため
同様のシステムを試作して実験を行なっていたが、
現在はNFCリーダつき{\SP}が普通のショップで売られている時代であるから、
技術的には完全に実用段階になっているといえる。

問題はユーザの思いこみと、インタラクションのアイデアの実装だけなのである

\section{サイネージ+実世界GUiで広がる世界}

実世界GUIの応用は無限である。
パソコン上ではあらゆる作業のためにGUIが利用されているのと同じように、
実世界におけるあらゆる作業に
{\SP}とSuicaを利用した実世界GUIを利用することができる。

普通の{\SP}とサイネージを利用して
例えば以下のようなサービスが考えられる。

\begin{itemize}
\item 買物案内

購入予定の商品を売っている店が近くに有るかどうかわかると便利である。
駅やショッピングモールのサイネージに{\SP}を近付けて操作することによって
そのような店の案内が表示されると良いであろう。
買いたい本や欲しいものを{\SP}で管理している人は多いと思われるので
このような機能は有用であろう。
大きな店や書店などでは売場の案内にも利用できる。

\item 行先案内

{\SP}で「乗換案内」のようなサービスで電車の経路を調べて移動することができるが、
乗換駅でどちらに歩いて行けばよいのか、
どの車輛に乗ると乗り換えが速いか、
などの情報を駅のサイネージで調べることができれば便利だろう。
この場合も、
{\SP}上で行先情報や経路はすでに入力されているのだから、
これを有効利用すればサイネージ上ではほとんど操作が不要にすることができるだろう。
{\SP}上のGoogleMapsで住所を調べた後で街角のサイネージに{\SP}をタッチすることにより
道案内を見るということもできるだろう。

\item 普通の情報検索

サイネージ上で{\SP}を操作するとき、
自分の興味のある分野が優先的にメニューに出るようになっていれば
検索が簡単になる。
テキストを使って情報検索を行なう場合でも、
{\SP}上での以前の検索ワードや入力文字列をサイネージ上で利用できれば
検索効率が上がるはずである。

\end{itemize}

これらはごく簡単な例にすぎない。
{\SP}もデータベースも既に存在するので、
これらをうまく組み合わせるだけでかなりの可能性があるといえるだろう。

\section{実世界GUI実現のインフラ}

サイネージと{\SP}を組み合わせることによって
実世界GUIによる有用なサービスが可能であることは確かであるが、
何もないところからGUIを構築するのは効率的でないため、
我々はGoldFishというインフラを作っている。

GoldFishはAndroidのアプリケーションで、
SuicaのようなRFIDを読んだ後でブラウザを起動し、
IDに対応づけられたURLにジャンプするという機能を持っている。
Androidのセンサをブラウザから利用できるようにするため、
JavaScriptからセンサ情報を読むことができるように設定した後で
ブラウザの起動を行なう。

例えば、
RFIDに対応づけられたページの中のJavaScriptで
Androidの回転を読み取って音量を制御するようなコードを書いておけば、
AndroidをRFIDに近付けてから回転するだけで
音量を制御できることになる。
このような実世界GUIのはサーバ上のJavaScriptで実行されるので、
端末に様々な実世界GUIプログラムをインストールしておく必要はなく、
あらかじめGoldFishだけをインストールしておけばよいことになる。
つまり、RFIDを貼ったポスタとサーバ上のプログラムだけ用意しておけば
自由自在に実世界GUIやそれを利用したサービスを提供できることになる。

\section{利用のシナリオ}

GoldFishを利用すると、
Web上のJavaScriptプログラムを書くだけで実世界GUIを実現することができ、
特殊なサーバを用意したりする必要が無い。
このため、
サイネージのように大規模に運用する場合でなくても
個人的に簡単に利用することが可能である。

例えば以下のような用途に利用できる。

\begin{itemize}
\item 家電の制御

RFIDに{\SP}をタッチすることによって家電をリモートコントロールする

\item 情報の添付

家電製品や本などにRFIDタグを貼っておき、
それに{\SP}でタッチすることにより
マニュアルや関連情報を調べられるようにする

\item 鍵のように

ドアに貼ったRFIDに{\SP}を近付けて回すことにより
ドアを開けられるようにする

\item データのコピペ

複数のパソコンにRFIDを貼っておき、
それに{\SP}を近付けて動かすことにより
データをコピーしたりペーストしたりできるようにする

\item 情報共有

掲示板にRFIDを貼っておき、
{\SP}でメッセージを貼り付けたりメッセージを読んだりする

\item 実世界ゲーム

いろんな場所にRFIDを貼っておき、オリエンテーリングや
宝捜しをして遊ぶ

\end{itemize}


GoldFIshの枠組みやRFIDリーダ入りケータイはまだ出たところであるが
その可能性は無限であり、
「パソコンを何に使うのか」と聞いてるみたいなものだといえるだろう。

早目にこういう可能性を考えることが大事だと思っている。





ちなみにこれは技術的な問題ではない。

イディオムがまだ発明されていないことと
ユーザの意識がそれに対応していないことが原因である

ユーザが「そもそも」何をしたいかを簡単にサイネージ上で表現することができれば話は全然かわる






\scriptsize
\bibliographystyle{plain}
\bibliography{article}

\end{document}
