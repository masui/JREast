\documentclass[a4j,11pt,twocolumn]{jarticle}
% \usepackage{graphicx,layout,prosym2010}
% \usepackage{here}

\makeatletter

\def\section{\@startsection {section}{1}{\z@}{-3.5ex plus -1ex minus -.2ex}{2.3ex plus .2ex}{\normalsize\bf}}

\def\@maketitle{
  \newpage\null\vskip 1em%
  \begin{center}%
	\let\footnote\thanks
	{\Large \bf \@jtitle \par}
	{\Large \bf \@etitle \par}
	\vskip 1.5em
 	{\large \bf \begin{tabular}[t]{c}\@jauthor \end{tabular}\par}
	\vskip 1.0em
	%\begin{tabular}[t]{c} \@jcontact \end{tabular}\par
	%\vskip 1.5em%
	%{\unvbox\@jabstractbox}
	%\vskip2em
% English title
	%{\Large \bf \@etitle \par}
	%\vskip 1.5em
	{\large \begin{tabular}[t]{c}\@eauthor \end{tabular}\par}
	\vskip .5em
	{\small \begin{tabular}[t]{c}\@econtact \end{tabular}\par}
	\vskip 1.5em
	{\unvbox\@eabstractbox}\par
	\par \vskip .5em
  \end{center}
  {\bf キーワード: \@jkeywords}\par
  {\bf\sf Keywords: \@ekeywords}
  \par \vskip 1.5em
}
%%%%%%%%%%%%%%%%%%%%%%%%%%%%%%%%%%%%%%%%%%%%%%%%%
%
%	Definitions
%
% Japanese & English titles
%
\def\jtitle#1{\gdef\@jtitle{#1}}
\def\@jtitle#{\mkt\@warning\jtitle}
\def\etitle#1{\gdef\@etitle{#1}}
\def\@etitle{\mkt@warning\etitle}

\def\jkeywords#1{\gdef\@jkeywords{#1}}
\def\@jkeywords#{\mkt\@warning\jkeywords}
\def\ekeywords#1{\gdef\@ekeywords{#1}}
\def\@ekeywords{\mkt@warning\ekeywords}
%
% Japanese & English authors
%
\long\def\jauthor#1{\long\gdef\@jauthor{#1}}
\def\@jauthor{\mkt@warning\jauthor}
\long\def\eauthor#1{\long\gdef\@eauthor{#1}}
\def\@eauthor{\mkt@warning\eauthor}
%
%  Japanese & English contact address
%
\long\def\jcontact#1{\long\gdef\@jcontact{#1}}
\def\@jcontact{\mkt@warning\jcontact}
\long\def\econtact#1{\long\gdef\@econtact{#1}}
\def\@econtact{\mkt@warning\econtact}

% Japanese & English abstracts

\def\jabstractname{あらまし}
\def\eabstractname{}

\newbox\@jabstractbox \global\setbox\@jabstractbox\box\voidb@x
\newbox\@eabstractbox \global\setbox\@eabstractbox\box\voidb@x

\def\jabstract{%
   \global\setbox\@jabstractbox\vbox\bgroup
      \hsize\textwidth \@parboxrestore
      {\bfseries \jabstractname}  \parindent=1zw\relax}
\def\endjabstract{\egroup}

\def\eabstract{%
      \small
   \global\setbox\@eabstractbox\vbox\bgroup
      \hsize\textwidth \@parboxrestore
      {\bfseries \eabstractname}  \parindent=1zw\relax}
\def\endeabstract{\egroup}
%%%%%%%%%%%%%%%%%%%%%%%


\makeatother


\pagestyle{empty}

\long\def\comment#1{}

\begin{document}

\def\figwidth{7.5cm}

\jtitle{デジタルサイネージと実世界GUI}
\jauthor{増井 俊之 (慶應義塾大学 環境情報学部)}
\jkeywords{実世界インタフェース, GUI, RFID}
\ekeywords{a, b, c}
\begin{eabstract}
Although many digital signage systems are set up in train stations and
other public spaces, they are usually used for showing general information
for the people who pass by, 
and they don't provide detailed information for individual users.
We propose ``real-world GUI'' interface where
\end{eabstract}

\etitle{Digital Signage and Real-world IF}
\eauthor{Toshiyuki Masui}
\econtact{Faculty of Environment and Information Studies, Keio University}

\maketitle

\section{デジタルサイネージと実世界GUI}

駅や街角でデジタルサイネージを見かける機会が多くなってきた。
遅延のような情報の伝達や広告のためにサイネージは効果的に利用されているが、
その可能性が充分に生かされているとはいえないのが現状である。

新規で複雑な機械を利用するのは難しい。
カードを改札機にタッチすれば扉が開くという
単純なインタフェースを持つSuicaの場合ですら
使い方を誰もが理解するようになるまでには時間がかかったことを考えると
誰もがサイネージを能動的に活用するようになるのは困難だと思われ、
現在のサイネージのほとんどはテレビのように情報を提供するだけにとどまっている。
サイネージのインタフェースはパソコンや携帯電話や家電機器のインタフェースと
異なるのが普通なので、検索のような機能が用意されているサイネージであっても
利用者は簡単に操作することはできないし、
長い間立ち止まって操作に習熟しようとする利用者もほとんどいないと思われる。

また、個々のユーザに対応した情報のやりとりが全くできないことも問題である。
昔の駅には「掲示板」が用意されているのが普通であり、
駅の利用者同志の非同期的なコミュニケーションに活用されていた。
現在は携帯電話が普及したためか駅の掲示板のほとんどは撤去されてしまっているが、
駅という特定の場所において時間差のあるコミュニケーションを行なうことができたという点で
掲示板は有用なものであった。
現在のデジタルサイネージを利用して同様のサービスを提供することは充分可能なはずであるが、
そのような目的のためにサイネージは現在利用されていない。
利用者が能動的的なコミュニケーションを行なわない場合でも、
利用者にとって有益な情報をサイネージで個別に提供することができれば有益なはずであるが、
現在のサイネージは
個々のユーザに対応した特別な要求や検索に答えるような工夫がされていないのが現状である。
ショッピングモールのサイネージでマップを見ることはできるかもしれないが、
利用者の性別や年齢により表示を変えることすら行なわれていない。

現在のサイネージは
設置場所固有の情報とインターネット上の通常の情報の両方を扱うことができるので、
インターネット上の情報にしかアクセスできない携帯電話のような装置を利用する場合よりも
有益な情報を提供できてしかるべきであるが、
操作の難しさのため、
利用者はもっぱら個人のパソコンや携帯電話を利用していると思われる。

サイネージを利用するとき、
簡単な操作で個人的な情報を少しでも利用することができれば
利用価値が大きく変わる可能性がある。
これはハードウェアやインフラの問題ではなく、
純粋にシステムの工夫が足りないことに起因している。
個人適応の機能及び使いなれた利用法を提供することが重要である。

長い時間をかけてパソコンのインタフェースがポピュラーになったのに比べると
サイネージはまだ歴史が浅いためユーザが戸惑うのは仕方がないともいえるが、
パソコンや携帯電話上の知見や新しいセンサなどを工夫することによって、
このような問題は一気に解決する可能性がある。

\section{実世界GUI}

完全に新規なシステムの使い方に習熟するのは難しいが、
パソコンのインタフェースのように
既に習熟しているシステムに似たものであれば簡単に慣れることができるかもしれない。

パソコンは非常に複雑な機械であるが、
MacやWindowsやLinuxは
長年の共進化のために同じような操作性をもつようになっているため、
プルダウンメニューやスクロールバーの使い方に戸惑うユーザは少ないはずである。
一方、こういうものを利用しにくいサイネージでは
はじめてのユーザはどう使えばよいのかわからないうえに練習する機会も少ないため、
複雑な機能を用意することは難しいと思われる。

近年、パソコンやブラウザはかなり普及しており、
多くの人々はこれらに習熟しているため、
サイネージの使い方がこれらに似ていれば
多くの人がサイネージを簡単に使いこなせるようになる可能性がある。

極端にいえば、パソコンと全く同じ使い方でサイネージを利用できるのであれば
多くの人が今すぐサイネージを使うようになるであろう。
しかし、駅や街角に設置するサイネージでは
パソコンと同じようなキーボードやマウスを利用することが難しいし、
そもそもパソコンは誰もがログインして使うようには設計されていない。
パソコンの利用法に似て異なる、
サイネージ用のインタフェース装置とインタフェース手法があれば良いであろう。

現在のパソコンは
メニュー、アイコン、スクロールバーのような
グラフィカルユーザインタフェース(GUI)で操作するのが普通であり、
多くの人々はこれらの操作に慣れている。
サイネージにおいてもこのようなGUI部品を利用することができれば
さほど混乱なくサイネージを活用できるようになると考えられる。
公共のサイネージにおいて、家でパソコンを扱うのと同じようにブラウザや地図を操作できたら
便利であろう。
たとえば、自分の地図や自分のブックマークがサイネージですぐに見られるならとても便利だろう。

パソコン上ではキーボードやマウスを使ってGUI部品を制御するが、
サイネージではこのような入力装置のかわりに
個人が持っているスマートフォンを利用することが考えられる。
スマートフォンは今後さらなる普及が期待されるうえに、
スマートフォンは傾きセンサや動きセンサを内蔵しているので
マウスのように利用することができる。
また、スマートフォンは独自のIDとデータを持っているため、
操作者の嗜好を反映した操作を行なうのも容易である。

サイネージ上でスマートフォンを活用する「実世界GUI」によって
サイネージの利用を劇的に改善することができるようになると考えられる。

\section{実世界GUIの実現}

パソコンのマウスは単純な相対移動検出装置であり、
それを活用して利用するGUIがパソコン上で工夫されているため、
実世界でGUIに似た操作を行なうことは難しくない。

(1)マウスの移動量にともなってGUIのマウスカーソルが動くようになっていて、
(2)マウスカーソルの下にボタンやメニューがあることを判定することができれば、
パソコン上のほとんどのGUIを実現することができるのであるから、
これと同じことを実世界で実現できればよいことになる。

例えば、実世界においてマウスのかわりにSuicaを利用することができる。
(1)ポスターの裏にSuicaリーダを内蔵しておき、それにSuicaをかざすことによってIDを読み取り、
(2)Suicaの動きを読みとる装置を用意しておけば、
Suicaをポスターにかざして動かすことによって
Suicaをマウスのように利用することができる。

マウスのかわりにスマートフォンを使う場合、
(1)スマートフォンを何かにタッチしたときIDを検出が可能であり、
(2)スマートフォンの動きを検出することができれば、
スマートフォンを実世界でマウスのように利用することができる。

沢山のSuicaリーダをポスターの裏に設置するのは実際的ではないが、
しかし最近のドコモなどから発売されている
Suicaを読み取り可能なNFCリーダ入りケータイを使うとこの問題が解決される。
たとえば、
「曲選択」と書いたパネルを用意しておき、
その裏側にSuicaを貼っておけば、
(1)NFCリーダを内蔵したスマートフォンで曲選択パネルにタッチすることによってSuicaのIDを読み取った後で、
(2)スマートフォンを移動すれば、
IDと移動量をもとにしてメニューのようにスマートフォンで選曲を行なうことができる。
また、このときスマートフォンがユーザの嗜好を知っていれば、
ユーザごとに異なる曲を選曲することができる。

我々はこういうシステムを長年研究している。
前者のようなシステムを「FieldMouse」\cite{}、
後者のようなシステムを「MouseField」\cite{}と呼んでおり、
様々な利用法を提案している。
10年ほど前この研究を開始したときは
まだNFCリーダつきスマートフォンが存在しなかったため
同様のシステムを試作して実験を行なっていたが、
現在はNFCリーダつきスマートフォンが普通のショップで売られている時代であるから、
技術的には完全に実用段階になっているといえる。

問題はユーザの思いこみと、インタラクションのアイデアの実装だけなのである

\section{サイネージ+実世界GUiで広がる世界}

実世界GUIの応用は無限である。
パソコン上ではあらゆる作業のためにGUIが利用されているのと同じように、
実世界におけるあらゆる作業に
スマートフォンとSuicaを利用した実世界GUIを利用することができる。

普通のスマートフォンとサイネージを利用して
例えば以下のようなサービスが考えられる。

\begin{itemize}
\item 買物案内

購入予定の商品を売っている店が近くに有るかどうかわかると便利である。
駅やショッピングモールのサイネージにスマートフォンを近付けて操作することによって
そのような店の案内が表示されると良いであろう。
買いたい本や欲しいものをスマートフォンで管理している人は多いと思われるので
このような機能は有用であろう。
大きな店や書店などでは売場の案内にも利用できる。

\item 行先案内

スマートフォンで「乗換案内」のようなサービスで電車の経路を調べて移動することができるが、
乗換駅でどちらに歩いて行けばよいのか、
どの車輛に乗ると乗り換えが速いか、
などの情報を駅のサイネージで調べることができれば便利だろう。
この場合も、
スマートフォン上で行先情報や経路はすでに入力されているのだから、
これを有効利用すればサイネージ上ではほとんど操作が不要にすることができるだろう。
スマホ上のGoogleMapsで住所を調べた後で街角のサイネージにスマホをタッチすることにより
道案内を見るということもできるだろう。

\item 普通の情報検索

サイネージ上でスマホを操作するとき、
自分の興味のある分野が優先的にメニューに出るようになっていれば
検索が簡単になる。
テキストを使って情報検索を行なう場合でも、
スマホ上での以前の検索ワードや入力文字列をサイネージ上で利用できれば
検索効率が上がるはずである。

\end{itemize}

これらはごく簡単な例にすぎない。
スマートフォンもデータベースも既に存在するので、
これらをうまく組み合わせるだけでかなりの可能性があるといえるだろう。

\section{実世界GUI実現のインフラ}

サイネージとスマートフォンを組み合わせることによって
実世界GUIによる有用なサービスが可能であることは確かであるが、
何もないところからGUIを構築するのは効率的でないため、
我々はGoldFishというインフラを作っている。

GoldFishはAndroidのアプリケーションで、
SuicaのようなRFIDを読んだ後でブラウザを起動し、
IDに対応づけられたURLにジャンプするという機能を持っている。
Androidのセンサをブラウザから利用できるようにするため、
JavaScriptからセンサ情報を読むことができるように設定した後で
ブラウザの起動を行なう。

例えば、
RFIDに対応づけられたページの中のJavaScriptで
Androidの回転を読み取って音量を制御するようなコードを書いておけば、
AndroidをRFIDに近付けてから回転するだけで
音量を制御できることになる。
このような実世界GUIのはサーバ上のJavaScriptで実行されるので、
端末に様々な実世界GUIプログラムをインストールしておく必要はなく、
あらかじめGoldFishだけをインストールしておけばよいことになる。
つまり、RFIDを貼ったポスタとサーバ上のプログラムだけ用意しておけば
自由自在に実世界GUIやそれを利用したサービスを提供できることになる。

\section{利用のシナリオ}

GoldFishを利用すると、
Web上のJavaScriptプログラムを書くだけで実世界GUIを実現することができ、
特殊なサーバを用意したりする必要が無い。
このため、
サイネージのように大規模に運用する場合でなくても
個人的に簡単に利用することが可能である。

例えば以下のような用途に利用できる。

\begin{itemize}
\item 家電の制御

RFIDにスマートフォンをタッチすることによって家電をリモートコントロールする

\item 情報の添付

家電製品や本などにRFIDタグを貼っておき、
それにスマートフォンでタッチすることにより
マニュアルや関連情報を調べられるようにする

\item 鍵のように

ドアに貼ったRFIDにスマートフォンを近付けて回すことにより
ドアを開けられるようにする

\item データのコピペ

複数のパソコンにRFIDを貼っておき、
それにスマートフォンを近付けて動かすことにより
データをコピーしたりペーストしたりできるようにする

\item 情報共有

掲示板にRFIDを貼っておき、
スマートフォンでメッセージを貼り付けたりメッセージを読んだりする

\item 実世界ゲーム

いろんな場所にRFIDを貼っておき、オリエンテーリングや
宝捜しをして遊ぶ

\end{itemize}


GoldFIshの枠組みやRFIDリーダ入りケータイはまだ出たところであるが
その可能性は無限であり、
「パソコンを何に使うのか」と聞いてるみたいなものだといえるだろう。

早目にこういう可能性を考えることが大事だと思っている。





ちなみにこれは技術的な問題ではない。

イディオムがまだ発明されていないことと
ユーザの意識がそれに対応していないことが原因である

ユーザが「そもそも」何をしたいかを簡単にサイネージ上で表現することができれば話は全然かわる









\end{document}
