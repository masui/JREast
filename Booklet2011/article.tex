■ デジタルサイネージと実世界GUI

駅や街角でデジタルサイネージを見かける機会が多くなってきた。
遅延のような情報の伝達や広告のためにサイネージは効果的に利用されているが、
その可能性が充分に生かされているとはいえないのが現状である。

特に、個々のユーザに対応した情報のやりとりが全くできないのは大きな問題である。
昔の駅には「掲示板」が用意されているのが普通であり、
駅の利用者同志の非同期的なコミュニケーションに活用されていた。
現在は携帯電話が普及したためか駅の掲示板のほとんどは撤去されてしまっているが、
駅という特定の場所において時間差のあるコミュニケーションを行なうことができたという点で
掲示板は有用なものであった。
現在のデジタルサイネージを利用して同様のサービスを提供することは充分可能なはずであるが、
そのような目的のためにサイネージは現在利用されていない。

利用者が能動的的なコミュニケーションを行なわない場合でも、
利用者にとって有益な情報をサイネージで個別に提供することができれば有益なはずであるが、
現在のサイネージはテレビと同じように誰にでも同じ情報を提供しているだけであり、
特別な要求や検索に答えるような工夫がされていないのが現状である。
ショッピングモールのサイネージでマップを見ることはできるかもしれないが、
利用者の性別や年齢により表示を変えることすら行なわれていない。
サイネージを利用するとき、個人的な情報を少しでも利用することができれば
利用価値が大きく変わる可能性がある。

「個人適応」と、「使いなれた利用法」を提供することが重要であろう

サイネージのインタフェースはパソコンや携帯電話や家電機器のインタフェースと
全く異なるのが普通なので、検索のような機能が用意されているサイネージであっても
利用者は簡単に操作することはできないし、
長い間立ち止まって操作に習熟しようとする利用者もほとんどいないと思われる。
現在のサイネージは
設置場所固有の情報とインターネット上の通常の情報の両方を扱うことができるので、
インターネット上の情報にしかアクセスできない携帯電話のような装置を利用する場合よりも
有益な情報を提供できてしかるべきであるが、
操作の難しさのため、
利用者はもっぱら個人のパソコンや携帯電話を利用しているのが現状である。

このように、現在サイネージが有効活用されていないのは、
操作性と個別ユーザ対応が充分でないからであると考えられる。
これはハードウェアやインフラの問題ではなく、
純粋にシステムの工夫が足りないことに起因している。
長い時間をかけてパソコンのインタフェースがポピュラーになったのに比べると
サイネージはまだ歴史が浅いためこれは仕方がない面もあるが、
パソコンや携帯電話上の知見や新しいセンサなどを工夫することによって
このような問題は一気に解決する可能性がある。

新規で複雑な機械を利用するのは難しい。
パソコンは非常に複雑な機械であるが、
MacやWindowsやLinuxは
長年の共進化のために同じような操作性をもつようになっているため、
プルダウンメニューやスクロールバーの使い方に戸惑うユーザはかなり少ないと思われる。
一方、こういうものを利用しにくいサイネージでは
一見のユーザはどう使えばよいのかわからないうえに練習する機会も少ないため、
複雑な機能を用意することは難しいと思われる。
カードを改札機にタッチすれば扉が開く、というしごく簡単なインタフェースを持つ
Suicaの場合ですら使い方を誰もが理解するようになるまでには時間がかかったことを考えると
誰もがサイネージを活用するようになるのは相当困難だと考えるのは普通であろう。

■ 実世界GUI

完全に新規なシステムに習熟するのは難しいが、
既に習熟しているシステムに似たものであれば簡単に慣れることができるかもしれない。

近年、パソコンやブラウザはかなり普及しており、
多くの人々はこれらに習熟しているため、
サイネージの使い方がこれらに似ていれば
多くの人がサイネージを簡単に使いこなせるようになる可能性がある。

極端にいえば、パソコンと全く同じ使い方でサイネージを利用できるのであれば
サイネージの利用は進むかもしれない。
しかし、駅や街角に設置するサイネージでは
パソコンと同じようなキーボードやマウスを利用することが難しいし、
そもそもパソコンは誰もがログインして使うようには設計されていない。
パソコンの利用法に似て異なる、サイネージ用のインタフェース装置とインタフェース手法があれば良いであろう。

現在のパソコンは
メニュー、アイコン、スクロールバーのような
グラフィカルユーザインタフェース(GUI)で操作するのが普通であり、
多くの人々はこれらの操作に慣れている。
サイネージにおいてもこのようなGUI部品を利用することができれば
さほど混乱なくサイネージを活用できるようになると考えられる。

公共のサイネージにおいて、家でパソコンを扱うのと同じようにブラウザや地図を操作できたら
便利なことだろう。

パソコン上ではキーボードやマウスを使ってGUI部品を制御するが、
サイネージではこのような入力装置のかわりに
個人が持っているスマートフォンを利用することが得策と考えられる。
現在多くの人間がスマートフォンを持ち歩いているうえに、
スマートフォンは傾きセンサや動きセンサを内蔵しているので
マウスのように利用することができる。
また、スマートフォンは独自のIDを持っているため
操作者の嗜好を反映した操作を行なうのも容易である。

たとえば、自分の地図や自分のブックマークがサイネージですぐに見られるならとても便利だろう。

サイネージ上でスマートフォンを活用する「実世界GUI」によって
サイネージの利用を劇的に改善することができるようになると考えられる。

■ 実世界GUIの実現


実はこれは簡単にできる。
Suicaやスマホを使えばいいのである。

SuicaにはRFIDが入ってありから、Suicaを認識できるサイネージの上でSuicaを動かせば
マウスのように使えるはずだし、
SuicaのIDから個人の嗜好などを利用できる


しかしSuicaリーダをあちこちに置くのは実際的ではない
あらゆるポスターにリーダを置くのは大変だろう
しかし最近のドコモなどのRFIDリーダ入りケータイを使うとこの問題が完全に解決される。

ポスターやサイネージの方にSUicaやRFIDを貼っておき、
スマホでそれにタッチした後で様々な操作を行なうようにすればよい。



我々はこういうシステムを長年研究している。
10年前はただの研究だったが、現在はインフラが整ってきたので技術的には完全に実用段階になっているといえる。

問題はユーザの思いこみと、インタラクションのアイデアの実装だけなのである


■ サイネージ+実世界GUiで広がる世界

どんなことができるか列挙してみたい。

* 地図を見る
  ルートを教えてくれる
  そのルートに問題がないか、とか
  これは個人情報がわからないと案内不可能である
* 自分の趣味のものが近くにあるかどうか
  買おうと思ってた本が駅の本屋にあることがわかる
* 好きなジャンルのレストランを調べる
  ついでにその状況もわかる
* 普通にググる
  ググるにはキーボードが必要と思われるかもしれないがそうでもない

■ ソフトウェアのインフラ

我々はGoldFishというインフラを作っている。

GOldFishはAndroidのアプリケーションで、
読んだIDのURLにジャンプするという単純なものである

しかし、Androidのセンサを利用することができるので
以下のようなことができる

---
---

これを利用するためには、サーバにJSを用意するだけでよい。
Android端末でRFIDにタッチすると
サーバ上のJavaScriptが端末上で動くわけだから、
端末に様々なプログラムをインストールしておく必要はないのである。


つまり、RFIDを貼ったポスタとサーバ上のプログラムだけ用意しておけば
自由自在に実世界GUIやそれを利用したサービスを提供できることになる。

■ 

GoldFIshの枠組みやRFIDリーダ入りケータイはまだ出たところであるが
その可能性は無限であり、
「パソコンを何に使うのか」と聞いてるみたいなものだといえるだろう。

早目にこういう可能性を考えることが大事だと思っている。









ちなみにこれは技術的な問題ではない。

イディオムがまだ発明されていないことと
ユーザの意識がそれに対応していないことが原因である

ユーザが「そもそも」何をしたいかを簡単にサイネージ上で表現することができれば話は全然かわる









